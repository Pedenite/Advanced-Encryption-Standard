\documentclass[12pt]{article}
\usepackage[brazil]{babel}
\usepackage[a4paper, total={6.5in, 9.5in}]{geometry}
\usepackage[utf8]{inputenc}

\usepackage{authblk}
\usepackage{lipsum}
\usepackage{xcolor}
\usepackage{listings}
\usepackage[normalem]{ulem}
\usepackage{amssymb}
\usepackage{float}
\usepackage{graphicx}
\usepackage{amsmath}
\usepackage{amsthm}
\usepackage{amssymb}
\newtheorem{theorem}{Theorem}[section]
\newtheorem{lemma}[theorem]{Lemma}

\definecolor{codegreen}{rgb}{0,0.6,0}
\definecolor{codegray}{rgb}{0.5,0.5,0.5}
\definecolor{codered}{rgb}{0.8,0,0}
\definecolor{backcolour}{rgb}{0.95,0.95,0.92}

\usepackage{inconsolata}
\lstset{
    language=python,
    backgroundcolor=\color{backcolour},   
    commentstyle=\color{codegreen},
    keywordstyle=\color{blue},
    numberstyle=\tiny\color{codegray},
    stringstyle=\color{codered},
    basicstyle=\ttfamily\small,
    numberstyle=\footnotesize,
    numbers=left,
    backgroundcolor=\color{white},
    frame=single,
    tabsize=2,
    rulecolor=\color{white},
    title=\lstname,
    escapeinside={\%*}{*)},
    breaklines=true,
    breakatwhitespace=true,
    framextopmargin=2pt,
    framexbottommargin=2pt,
    inputencoding=utf8,
    extendedchars=true,
    showstringspaces=false,
    literate={á}{{\'a}}1 {ã}{{\~a}}1 {é}{{\'e}}1 {Ó}{{\'O}}1 {Ã}{{\~A}}1 {í}{{\'i}}1 {ó}{{\'o}}1 {ç}{{\.c}}1 {ê}{{\^e}}1 {ú}{{\'u}}1,
}

% \pagecolor[rgb]{0.1,0.1,0.1} %black
% \color[rgb]{0.75,0.75,0.75} %grey

\title{\textbf{Segurança Computacional\\ \Large{Advanced Encryption Standard}}}
\author{Pedro Henrique de Brito Agnes, 18/0026305 \\
Pedro Pessoa Ramos, 18/0026488}
\affil{Dep. Ciência da Computação - Universidade de Brasília (UnB) \vspace{-2ex}}
\date{}

\begin{document}
\maketitle

\section{Implementação do AES}
A implementação do AES foi feita fixa para a versão de 128 bits e para isso foi criado o arquivo \texttt{aes.py} na pasta \texttt{src} usando a linguagem python preferencialmente na versão 3.6 ou acima. Para executar o programa para cifrar o arquivo \texttt{sample/dupla.jpg}, por exemplo com uma chave pseudo-aleatória usando o padrão de \textit{rounds} no modo ECB e salvar o criptograma no arquivo \texttt{sample/10r.jpg}, deve ser usado o seguinte comando:

\begin{lstlisting}
python src/aes.py sample/dupla.jpg -o sample/10r.jpg
\end{lstlisting}

O primeiro argumento que o programa recebe é o arquivo que contém a mensagem. Logo em seguida, está sendo passado o argumento \texttt{-o}, que é obrigatório e representa o arquivo onde será impresso o criptograma. Como não foi passada uma chave pré-existente, será gerada a chave durante a execução e ela será salva na pasta \texttt{keys} e será impressa uma mensagem informando o nome exato do arquivo que a contém. Existem outros argumentos opcionais que podem ser listados com o \texttt{-h}:

\begin{lstlisting}
python src/aes.py -h
\end{lstlisting}

Segue a lista de argumentos aceitos:
\begin{itemize}
    \item \textbf{-k} - Arquivo com a chave para criptografar/descriptografar. Argumento obrigatório se for acionada a opção para descriptografar.
    \item \textbf{-o} - Arquivo onde será feito o output. Obrigatório.
    \item \textbf{-r} - Número positivo que representa a quantidade de \textit{rounds} que o AES irá usar. Se não passado um valor, será usado o padrão de 10.
    \item \textbf{-d} - Argumento que indica que o programa vai descriptografar. Deve ser passado no final do comando sem parâmetros adicionais.
\end{itemize}

Portanto, como exemplo, para cifrar o mesmo arquivo do exemplo acima usando a mesma chave (ex.: \texttt{keys/key\_1}) com 1 \textit{round} e colocando o output no arquivo \texttt{sample/1r.jpg}, pode ser usado o seguinte comando:

\begin{lstlisting}
python src/aes.py sample/dupla.jpg -k keys/key_1 -r 1 -o sample/1r.jpg
\end{lstlisting}

Da mesma forma, para decifrá-lo no arquivo \texttt{sample/decoded.jpg} por exemplo, pode-se usar o comando abaixo:

\begin{lstlisting}
python src/aes.py sample/1r.jpg -k keys/key_1 -r 1 -o sample/decoded.jpg -d
\end{lstlisting}

\subsection{Aspectos Técnicos}
Para a implementação descrita acima, temos 2 arquivos principais que realizam as operações do AES, ambos na pasta \texttt{src/symmetric}. O \texttt{cipher} é usado para cifrar e o \texttt{decipher} é usado para decifrar. De forma resumida, o funcionamento da cifração se dá na seguinte forma:

\begin{lstlisting}
self.addRoundKey()          # realiza o XOR com a chave inicial
for i in range(rounds):     # itera sobre os rounds
    self.expandKey(i)       # transforma a chave
    self.subBytes()
    self.shiftRows()
    if i != rounds-1:       # não realiza o mixColumns no último round
        self.mixColumns()
    
    self.addRoundKey()      # Realiza o XOR com a chave da rodada
\end{lstlisting}

Já para a decifração, é feito o contrário das operações, sendo que algumas foram modificadas apesar de ter o mesmo nome e segue a seguinte rotina:

\begin{lstlisting}
for x in range(rounds):     # vai até a chave final obtida na cifração
    self.expandKey(x)

for i in range(rounds):     # itera sobre os rounds
    self.addRoundKey()      # realiza o XOR com a chave da rodada
    if i != 0:              # não realiza o mixColumns no round inicial
        self.mixColumns()

    self.shiftRows()
    self.subBytes()
    self.shrinkKey(rounds-1-i) # operação reversa da expandKey
    
self.addRoundKey()          # Realiza o XOR com a chave inicial
\end{lstlisting}

\end{document}
