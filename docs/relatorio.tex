\documentclass[12pt]{article}
\usepackage[brazil]{babel}
\usepackage[a4paper, total={6.5in, 9.5in}]{geometry}
\usepackage[utf8]{inputenc}

\usepackage{authblk}
\usepackage{lipsum}
\usepackage{xcolor}
\usepackage{listings}
\usepackage[normalem]{ulem}
\usepackage{amssymb}
\usepackage{float}
\usepackage{graphicx}
\usepackage{amsmath}
\usepackage{amsthm}
\usepackage{amssymb}
\newtheorem{theorem}{Theorem}[section]
\newtheorem{lemma}[theorem]{Lemma}

\definecolor{codegreen}{rgb}{0,0.6,0}
\definecolor{codegray}{rgb}{0.5,0.5,0.5}
\definecolor{codered}{rgb}{0.8,0,0}
\definecolor{backcolour}{rgb}{0.95,0.95,0.92}

\usepackage{inconsolata}
\lstset{
    language=sh,
    backgroundcolor=\color{backcolour},   
    commentstyle=\color{codegreen},
    keywordstyle=\color{blue},
    numberstyle=\tiny\color{codegray},
    stringstyle=\color{codered},
    basicstyle=\ttfamily\small,
    numberstyle=\footnotesize,
    numbers=left,
    backgroundcolor=\color{white},
    frame=single,
    tabsize=2,
    rulecolor=\color{white},
    title=\lstname,
    escapeinside={\%*}{*)},
    breaklines=true,
    breakatwhitespace=true,
    framextopmargin=2pt,
    framexbottommargin=2pt,
    inputencoding=utf8,
    extendedchars=true,
    showstringspaces=false,
    literate={á}{{\'a}}1 {ã}{{\~a}}1 {é}{{\'e}}1 {Ó}{{\'O}}1 {Ã}{{\~A}}1 {í}{{\'i}}1 {ó}{{\'o}}1 {ç}{{\.c}}1 {ê}{{\^e}}1,
}

% \pagecolor[rgb]{0.1,0.1,0.1} %black
% \color[rgb]{0.75,0.75,0.75} %grey

\title{\textbf{Segurança Computacional\\ \Large{Advanced Encryption Standard}}}
\author{Pedro Henrique de Brito Agnes, 18/0026305 \\
Pedro Pessoa Ramos, 18/0026488}
\affil{Dep. Ciência da Computação - Universidade de Brasília (UnB) \vspace{-2ex}}
\date{}

\begin{document}
\maketitle

\section{Implementação do AES}
A implementação do AES foi feita fixa para a versão de 128 bits e para isso foi criado o arquivo \texttt{aes.py} na pasta \texttt{src} usando a linguagem python preferencialmente na versão 3.6 ou acima. Para executar o programa para cifrar um arquivo \texttt{sample/teste.txt} por exemplo com uma chave aleatória usando o padrão de 10 \textit{rounds} e salvar o criptograma no arquivo sample/output.txt, deve ser usado o seguinte comando:

\begin{lstlisting}
python src/aes.py sample/teste.txt -o output.txt
\end{lstlisting}

O primeiro argumento que o programa recebe é o arquivo que contém a mensagem e o argumento \texttt{-o} é obrigatório e representa o arquivo onde será impresso o criptograma. Existem outros argumentos opcionais que podem ser listados com o \texttt{-h}:

\begin{lstlisting}
python src/aes.py -h
\end{lstlisting}

Segue a lista de argumentos aceitos:
\begin{itemize}
    \item \textbf{-k} - Arquivo com a chave para criptografar/descriptografar. Argumento obrigatório se for acionada a opção para descriptografar.
    \item \textbf{-o} - Arquivo onde será feito o output. Obrigatório.
    \item \textbf{-r} - Número positivo que representa a quantidade de \textit{rounds} que o AES irá usar. Se não passado um valor, será usado o padrão de 10.
    \item \textbf{-d} - Argumento que indica que o programa vai descriptografar. Deve ser passado no final sem parâmetros adicionais.
\end{itemize}

Portanto, como exemplo, para decifrar um criptograma no arquivo \texttt{sample/ex.txt} usando a chave \texttt{keys/key\_sample} e 3 \textit{rounds} e colocando o output no arquivo \texttt{sample/out.txt}, pode ser usado o seguinte comando:

\begin{lstlisting}
python src/aes.py sample/ex.txt -k keys/key_sample -r 3 -o sample/out.txt -d
\end{lstlisting}

\subsection{Aspectos Técnicos}


\end{document}
